--delot za latex
\documentclass[12pt]{article}
\usepackage[utf8]{inputenc}
\usepackage[T2A]{fontenc}
\usepackage{xcolor}
\usepackage{amsmath}
\usepackage{esint}
\begin{document}
\title{ТЕОРИЈА:}
За да филтрираме слика во фреквенциски домен потребно е да ја помножиме
сликата во фреквенциски домен со филтерот (кернел) кој е исто така во
фреквенциски домен.
\begin{equation}
I_{f,filt}=HI_f
\end{equation}
За таа цел потребно е да се вметне Octave функција која ќе креира “Band pass
gaussian filter” во фреквенциски домен. Bandpass гаусовиот филтер во
фреквенциски домен е дефиниран со равенката:
\begin{equation}
H_{(i,j)}=e^{-\frac{d(i,j)^2}{f_h^2}}(1-e^{-\frac{d(i,j)^2}{f_l^2}})
\end{equation}

Каде што $d(i,j)$ е евклидовото растојанието на даден пиксел од центарот на
сликата, додека $f_h$ и $f_l$ се праговите на високите и ниските фреквенции кои сакаме
да ги филтрираме.
Конверзија на 2D сликата од просторен во фреквенциски домен се прави со
користење на Фуриева трансформација. Фуриевата трансформација е дефинирана
како:

         $F(u,v)=\iint \limits_{-\infty}^{\infty} f(x,y)e^{-i2\pi(ux+vy)}dxdy$


А во дискретниот домен горната равенка може да биде преведена како
\begin{equation}
F(u,v)=\sum_{m=-\infty}^{\infty}\sum_{n=-\infty}^{\infty}f[m,n]\cdot e^{-i2\pi(xmu_0+ynv_0)}
\end{equation}
Конверзија на 2D сликата од фреквенциски во просторен домен се прави со
користење на инверзна Фуриева трансформација. Инверзната Фуриева
трансформација е дефинирана како:

$f(x,y)=\iint \limits_{-\infty}^{\infty}F(u,v)e^{i2\pi(ux+vy)}dudv$

А во дискретниот домен горната равенка може да биде преведена како
\begin{equation}
f(x,y)=\sum_{m=-\infty}^{\infty}\sum_{n=-\infty}^{\infty}F(m,n)\cdot e^{i2\pi(xmu_0+ynv_0)}
\end{equation}
\end{document}

--prikaz na slikite

%%file create_filter.m
function filter = create_filter(nx,ny,d0,d1)
filter = ones(nx,ny);
for i = 0:nx-1,
            for j = 0:ny-1
                 dist= sqrt((i-nx/2)^2 + (j-ny/2)^2);
                 filter(i+1,j+1) = exp(-dist^2/(2*(d1^2))).*(1.0-exp(-dist^2/(2*(d0^2))));
            end
end
filter(nx/2+1,ny/2+1)=1;
end

%store data
clear all
%read image
im = double(imread('ice.jpg'));
%size of the image
[nx ny] = size(im);
%Transform the image from spatial to frequency domain
im_fft = fftshift(fft2(im));
% Create Gaussian bandpass kernels, in frequency domain, for 20 different cut off frequencies and filter the image
p=1;
for i = 1:10:21
q=1;
    for j = 1:10:21
        %Create the bandpass kernel
        tmp1 = create_filter(nx,ny,i,j);
        %Filter the image with the bandpass kernel and transform it back to spatial domain
        tmp = (abs(ifft2(ifftshift(tmp1.*im_fft))));
        %Save the filtered image (in spatial domain) and kernel (in frequency domain) for visualization 
        filtered_image(p,q,:) = tmp(:);
        filter_kernel(p,q,:) = tmp1(:);
        q=q+1;
    end
    
    p=p+1;
end
save filtered_image
save filter_kernel
imshow(tmp) %to see the final result of the picture tmp
imshow(tmp1) %to see the final result of the picture tmp1

#libraries that we will use in python
import ipywidgets as widgets
from IPython.display import display
import matplotlib.pyplot as plt
import numpy as np
import matplotlib.image as mpimg
import scipy.io as sio
import plotly.graph_objects as go
from chart_studio.plotly import plot, iplot as py
import cv2
%matplotlib inline

#saving the variables
filtered_kernel = sio.loadmat('filter_kernel.mat')
f_kernels = filtered_kernel['filter_kernel']
filtered_image = sio.loadmat('filtered_image.mat')
f_images = filtered_image['filtered_image']

#the 2 range slider
value=widgets.IntRangeSlider(
    value=[0, 10],
    min=0,
    max=10,
    step=1,
    disabled=False,
    continuous_update=False,
    orientation='horizontal',
    readout_format='d',
    layout={'width': '100%', 'height':'100%'}
)

def displayimage(first, second):
    f_images = np.rot90((f_images[first][second][0:].reshape(402,566)), -1)
    return f_images;
    

def displaykernel(first, second):
    f_kernels = np.rot90((f_kernels[first][second][0:].reshape(402,566)), -1)
    return f_kernels;

def upgradep(value):
    image = displayimage(value[0], value[1])
    kernel = displaykernel(value[0], value[1])
    
def display(value):
    plot_image = np.concatenate((cv2.flip(upgradep.image, 1), cv2.flip(upgradep.kernel, 1)))
    plt.figure(figsize=(20,30))
    trace = go.Heatmap(z=z, 
                   colorscale="gray")
{"data": [trace]}
fig = go.Figure(data=go.Data([trace]))
fig['layout'].update(
    title="Annotated Heatmap",
    xaxis=go.XAxis(ticks='', side='top'),
    yaxis=go.YAxis(ticks='', ticksuffix='  '), 
    width=402,
    height=566,
    autosize=False
)
print py.plot(fig, filename='Stack Overflow 31756636', auto_open=False)
    plt.axis('off')
    plt.show()
    
w=widgets.interactive(upgradep, value=value)
w

