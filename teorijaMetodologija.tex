
\documentclass[12pt]{article}
\usepackage[utf8]{inputenc}
\usepackage[T2A]{fontenc}
\usepackage{xcolor}
\usepackage{amsmath}
\usepackage{esint}
\begin{document}
\title{ТЕОРИЈА:}
За да филтрираме слика во фреквенциски домен потребно е да ја помножиме
сликата во фреквенциски домен со филтерот (кернел) кој е исто така во
фреквенциски домен.
\begin{equation}
I_{f,filt}=HI_f
\end{equation}
За таа цел потребно е да се вметне Octave функција која ќе креира “Band pass
gaussian filter” во фреквенциски домен. Bandpass гаусовиот филтер во
фреквенциски домен е дефиниран со равенката:
\begin{equation}
H_{(i,j)}=e^{-\frac{d(i,j)^2}{f_h^2}}(1-e^{-\frac{d(i,j)^2}{f_l^2}})
\end{equation}

Каде што $d(i,j)$ е евклидовото растојанието на даден пиксел од центарот на
сликата, додека $f_h$ и $f_l$ се праговите на високите и ниските фреквенции кои сакаме
да ги филтрираме.
Конверзија на 2D сликата од просторен во фреквенциски домен се прави со
користење на Фуриева трансформација. Фуриевата трансформација е дефинирана
како:

         $F(u,v)=\iint \limits_{-\infty}^{\infty} f(x,y)e^{-i2\pi(ux+vy)}dxdy$


А во дискретниот домен горната равенка може да биде преведена како
\begin{equation}
F(u,v)=\sum_{m=-\infty}^{\infty}\sum_{n=-\infty}^{\infty}f[m,n]\cdot e^{-i2\pi(xmu_0+ynv_0)}
\end{equation}
Конверзија на 2D сликата од фреквенциски во просторен домен се прави со
користење на инверзна Фуриева трансформација. Инверзната Фуриева
трансформација е дефинирана како:

$f(x,y)=\iint \limits_{-\infty}^{\infty}F(u,v)e^{i2\pi(ux+vy)}dudv$

А во дискретниот домен горната равенка може да биде преведена како
\begin{equation}
f(x,y)=\sum_{m=-\infty}^{\infty}\sum_{n=-\infty}^{\infty}F(m,n)\cdot e^{i2\pi(xmu_0+ynv_0)}
\end{equation}
\end{document}


--Python za filtriranje na slika
import cv2
import numpy as np
import matplotlib.pyplot as plot
from PIL import Image, ImageOps
%matplotlib inline

slika = cv2.imread('ice.jpg',0)
plot.imshow(img)

f = np.fft.fft2(slika)
fshift = np.fft.fftshift(f)
magnitude_spectrum = 20*np.log(np.abs(fshift))

plot.subplot(121),plot.imshow(slika, cmap = 'gray')
plot.title('Vnesena fotografija'), plot.xticks([]), plot.yticks([])
plot.subplot(122),plot.imshow(magnitude_spectrum, cmap = 'gray')
plot.title('Dobiena'), plot.xticks([]), plot.yticks([])
plot.show()

--So matlab koga sakame da manipulirame so slika
clear all
close all

a=imread('C:\Users\mlade\Desktop\ice.jpg');
whos a
g=fspecial('gaussian',960,10);
max(g(:))
g1=mat2gray(g);
max(g1(:))
af=fftshift(fft2(a));
ag1=af.*g1;
fftshow(ag1)

--Delot za Teorija go pisuvav vo Latex a probav da gi resavam posebno vo python i vo matlab poradi toa sto imav problem so matlab kernelot.
